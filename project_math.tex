\documentclass{article}
\usepackage{graphicx} % Required for inserting images
\usepackage{amsmath}  % For advanced math formatting
\usepackage{braket}   % For Dirac notation
\usepackage{physics}  % Optional: for common quantum physics notation
\usepackage{hyperref} % Optional: for clickable links
\usepackage{geometry} % Optional: to adjust page margins
\usepackage{minted}
\geometry{margin=1in}

\title{Chemistry Summer Project}
\author{pp583}
\date{July 2025}

\begin{document}

\maketitle

\section{Procedure for finding parameters to feed into a machine learning model}

\subsection*{1. Get Localized Molecular Orbitals}

\begin{itemize}
    \item Generate a molecule.

    \item Using \verb|mf.mo_occ| and \verb|mf.mo_coeff|, split the molecular orbital (MO) coefficients into those corresponding to occupied and virtual (unoccupied) orbitals.

    \item The molecular orbitals ($\psi_i$) are expressed as:
    \[
    \psi_i = \sum_{\mu=1}^{K} \phi_\mu C_{\mu i}
    \]
    where K is the number of atomic orbitals
    \item After splitting, we define:
    \[
    \psi_i^{\mathrm{occ}} = \sum_{\mu=1}^{K} \phi_\mu C_{\mu i}
    \qquad \text{and} \qquad
    \psi_a^{\mathrm{vir}} = \sum_{\mu=1}^{K} \phi_\mu C_{\mu a}
    \]
    where $i = 1,2...N_{occ}$ and $a = 1,2...N_{vir} $
    \item Then we localize the orbitals by finding a unitary rotation:
    \[
    \widetilde{\psi}_{\mathrm{occ}} = \psi_{\mathrm{occ}} \mathrm{U}_{\mathrm{occ}}
    \qquad \text{and} \qquad
    \widetilde{\psi}_{\mathrm{vir}} = \psi_{\mathrm{vir}} \mathrm{U}_{\mathrm{vir}}
    \]

    \item In our coordinate system, set by the basis functions, we represent this as:
    \[
    \widetilde{C}_{\mu I}^{\mathrm{occ}} = \sum_{i} C_{\mu i}^{\mathrm{occ}} U_{iI}
    \qquad \text{and} \qquad
    \widetilde{C}_{\mu A}^{\mathrm{vir}} = \sum_{a} C_{\mu a}^{\mathrm{vir}} U_{aA}
    \]

    \item Then we stack the two matrices to obtain the new matrix with the coefficients of the localized molecular orbitals: \( \widetilde{C}  = (\widetilde{C}_{\mu I}^{\mathrm{occ}}, \widetilde{C}_{\mu I}^{\mathrm{vir}})\)
\end{itemize}

\subsection*{2. Perform Population Analysis on the Molecular Orbitals}
\subsubsection*{\textit{\textbf{!}} usually we should orient orient all the bonds along the z-axis when performing these calculations but I am not sure how this will affect our orbitals, so for now I am working with diatomic molecules that are oriented along the z-axis}   

\begin{itemize}
    \item The number of electrons in each orbital is:
    \[
    n_i = \braket{\widetilde{\psi}_i} = 1
    \]

    \item In our chosen basis, this becomes:
    \[
    n_i = \sum_{\mu, \nu} \widetilde{C}_{\mu i}^{*} \widetilde{C}_{\nu i} \int \phi_{\mu}^{*}(r) \phi_{\nu}(r) \,dr = \sum_{\mu, \nu} \widetilde{C}_{\mu i}^{*} \widetilde{C}_{\nu i} S_{\mu \nu}
    \]

    \item We define the \textbf{atomic population} of orbital \( \psi_i \) on atom \( A \) as:
    \[
    q_A = \sum_{\mu \in A} \sum_{\nu} \widetilde{C}_{\mu i}^{*} \widetilde{C}_{\nu i} S_{\mu \nu} = \sum_{\mu \in A} \sum_{\nu} \widetilde{C}_{i \mu}^{\dagger} S_{\mu \nu} \widetilde{C}_{\nu i}
    \]

    \item By looking at \( q_A \) for the different atoms, we can determine which atoms the orbitals are centered on (along with knowing interatomic distances).
\end{itemize}

\subsection*{3. Calculate the Expectation Values of \( \hat{L}_z \) for the localized Molecular Orbitals}

\begin{itemize}
    \item The expectation value is:
    \[
    \bra{\widetilde{\psi}_i} {\hat{L}_z}\ket{\widetilde{\psi}_i} = \sum_{\mu,\nu} \widetilde{C}_{\mu i}^{*} \bra{\phi_\mu} \hat{L}_z \ket{\phi_\nu} \widetilde{C}_{\nu i} = \sum_{\mu,\nu} \widetilde{C}_{i \mu}^{\dagger} L_{\mu \nu} \widetilde{C}_{\nu i} = (\widetilde{C}^{\dagger} L \widetilde{C})_{ii}
    \]
    where \( L \) is the matrix representation of the angular momentum operator \(\hat{L}_z\) in the basis \( \{ \phi_\mu \} \) 

    However, since the matrix representation of $\hat{L}_z$ in a real-valued atomic orbital basis has vanishing diagonal elements, it is more appropriate to use the squared operator $\hat{L}_z^2$ to characterize the angular momentum of molecular orbitals.

    \[
    \bra{\widetilde{\psi}_i} {\hat{L}_z^{2}}\ket{\widetilde{\psi}_i} = \sum_{\mu,\nu} \widetilde{C}_{\mu i}^{*} \bra{\phi_\mu} \hat{L}_z^{2} \ket{\phi_\nu} \widetilde{C}_{\nu i} = \sum_{\mu,\nu} \widetilde{C}_{\mu i}^{*} \sum_{\lambda}( \bra{\phi_\mu} \hat{L}_z \ket{\phi_\lambda} \bra{\phi_\lambda} \hat{L}_z\ket{\phi_\nu}) \widetilde{C}_{\nu i} = \sum_{\mu,\nu} \widetilde{C}_{i \mu}^{\dagger} L_{\mu \nu}^2 \widetilde{C}_{\nu i} = (\widetilde{C}^{\dagger} L^2 \widetilde{C})_{ii}
    \]

    The \verb|pyscf| code for obtaining the $\hat{L}_z$ matrix is:

\begin{minted}[fontsize=\small, linenos]{python}
from pyscf.gto import moleintor
# Build molecule
mol = gto.Mole()
mol.atom = 'H 0 0 0; F 0 1.1 0'  
mol.basis = 'sto-3g'            
mol.build()

# Compute angular momentum operator integrals: r × p
lz_3comp = moleintor.getints('int1e_cg_irxp_sph',
                             mol._atm, mol._bas, mol._env,
                             comp=3)
lz_matrix = lz_3comp[2]

\end{minted}
\end{itemize}

\subsection*{4. Calculate the expectation values of \(\hat{F})\) for the localized Molecular orbitals(i.e estimate their energies)}

\begin{itemize}
    \item The expectation value is: 
    \[
    \bra{\widetilde{\psi}_i}\hat{F}\ket{\widetilde{\psi}_i} = \widetilde{\epsilon}_i
    \]

    \[
    \widetilde{\epsilon}_i = \int (\sum_{p} \psi_{p}^{*} U_{pi}^{*}) \hat{F}(\sum_{q} \psi_{q} U_{qi}) =
    \int (\sum_{p} \psi_{p}^{*} U_{pi}^{*})(\sum_{q} \epsilon_q \psi_{q} U_{qi}) =
    \sum_{pq} U_{pi}^{*}U_{qi}\epsilon_{q}\delta_{pq} = \sum_{pq} U_{ip}^{\dagger}U_{qi}\epsilon_{q}\delta_{pq} = \sum_{p} |U_{pi}|^{2}\epsilon_{p}
    \]
    this is a \textbf{weighted average} of the energies of the canonical MOs $\epsilon_p$
    U has the form:
    \[ U =  
    \begin{pmatrix}
    U^{occ} & 0 \\
    0 & U^{vir}
    \end{pmatrix}   
    \]

    if we turn $\epsilon_p$ into a diagonal matrix, then the above expression will have the following form:

    \[
    \widetilde{\epsilon}_i = (U^{\dagger}\epsilon U)_{ii}
    \]

\end{itemize}
\end{document}
